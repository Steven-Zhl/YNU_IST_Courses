% !TEX program = xelatex
\documentclass[hyperref,a4paper,UTF8]{ctexart}

\usepackage[left=2.50cm, right=2.50cm, top=2.50cm, bottom=2.50cm]{geometry}

\usepackage[unicode=true,colorlinks,urlcolor=blue,linkcolor=blue,bookmarksnumbered=true]{hyperref}
\usepackage{latexsym,amssymb,amsmath,amsbsy,amsopn,amstext,amsthm,amsxtra,color,bm,calc,ifpdf,booktabs}
\usepackage{graphicx}
\usepackage{enumerate}
\usepackage{fancyhdr}
\usepackage{listings}
\usepackage{multirow}
\usepackage{makeidx}
\usepackage{xcolor}
\usepackage{fontspec}
\usepackage{subfigure}
\usepackage{hyperref}
\usepackage{pythonhighlight}
\usepackage[ruled,linesnumbered]{algorithm2e}


\pagestyle{fancy}
\fancyhead[L]{}
\fancyhead[C]{\fangsong 云南大学信息学院机器学习实验(2023春)课程论文}
\fancyhead[R]{}

\renewcommand{\abstractname}{\textbf{\large {摘\quad 要}}} % 更改摘要二字的样式


\title{
机器学习实验(2023春)课程论文\\
~\\
\textbf{在此填写题目}
}
\author{
\kaishu\normalsize
姓名\ \underline{张三} \qquad
学号\ \underline{2020XXXXXXX} \qquad
}
\date{} % 留空,不显示日期


\begin{document}

\begin{figure}
    \centering
    \includegraphics[width=0.7\textwidth]{fig/ynu.jpg}
\end{figure}

\maketitle



\begin{abstract}

    在此填写摘要内容

\end{abstract}
\thispagestyle{empty} % 当前页不显示页码

\newpage

\tableofcontents
\thispagestyle{empty} % 当前页不显示页码

\newpage

\listoffigures
\listoftables


\thispagestyle{empty} % 当前页不显示页码
\newpage


\section{引言}

在此填写引言


\section{如何使用\LaTeX }

\subsection{文字加粗倾斜}

以下是一些样例:

\textbf{加粗文本}
不空行效果


\textit{倾斜文本}

\underline{下划线文本}

\subsection{项目编号}

\subsubsection{不带数字的项目编号}
\begin{itemize}
    \item XXX
    \item XXX
    \item XXX
\end{itemize}

\subsubsection{带数字的项目编号}
\begin{enumerate}
    \item XXX
    \item XXX
    \item XXX
\end{enumerate}

\subsection{数学公式}

\begin{itemize}
    \item 行内公式:$h_\theta(x) = \frac{1}{1 + e^{-\theta^Tx}}$
    \item 另起一行且带编号的公式:
          \begin{equation}
              J(\theta) = -\frac{1}{m}\sum_{i=1}^{m}[y^{(i)}\log(h_\theta(x^{(i)})) + (1-y^{(i)})\log(1-h_\theta(x^{(i)}))]
          \end{equation}

    \item 另起一行且不带编号的公式:
          \begin{equation*}
              J(\theta) = -\frac{1}{m}\sum_{i=1}^{m}[y^{(i)}\log(h_\theta(x^{(i)})) + (1-y^{(i)})\log(1-h_\theta(x^{(i)}))]
          \end{equation*}
\end{itemize}








\subsection{插入图片}
使用figure环境插入图片
\begin{itemize}
    \item 使用width控制图片宽度,0.7\textbackslash textwidth表示0.7倍页面宽度
    \item \{\} 内放图片路径,建议所有图片放在fig文件夹下
\end{itemize}


\begin{figure}[ht]
    \centering
    \includegraphics[width=0.5\textwidth]{fig/example.jpg}
    \caption{在此填写图片的标题}
    \label{fig:example}
\end{figure}


\subsection{插入表格}
使用table环境插入表格。

\begin{table}[ht]
    \centering
    \begin{tabular}{c|lr} % 通过添加 | 来表示是否需要绘制竖线
        \toprule
        表头 & col1 & col2 \\
        \midrule
        1  & 2    & 3    \\
        \hline  %在第一行和第二行之间绘制横线
        4  & 5    & 6    \\
        7  & 8    & 9    \\
        \bottomrule
    \end{tabular}
    \caption{请插入表格标题}
    \label{tab:example}
\end{table}







\subsection{引用与参考文献}

首先将参考文献的bibtex放入.bib文件,例如
\begin{verbatim}
    @book{Goodfellow-et-al-2016,
        title={Deep Learning},
        author={Ian Goodfellow and Yoshua Bengio and Aaron Courville},
        publisher={MIT Press},
        note={\url{http://www.deeplearningbook.org}},
        year={2016}
    }
\end{verbatim}



然后使用\textbackslash cite 引用.bib文件中的文献,如:
\textbackslash cite\{Goodfellow-et-al-2016\},将生成引用 \cite{Goodfellow-et-al-2016}

这是第二篇 \cite{DBLP:conf/alt/ChakrabartyL23}

\textit{*如何查找某篇论文的bibtex? 可以使用\url{https://dblp.org/}}

\subsection{算法伪代码}

使用\textbackslash algorithm2e环境插入算法伪代码


%% This is needed if you want to add comments in
%% your algorithm with \Comment
\SetKwComment{Comment}{/* }{ */}

\begin{algorithm}[h]
    \caption{An algorithm with caption}\label{alg:example}
    \KwData{$n \geq 0$}
    \KwResult{$y = x^n$}
    $y \gets 1$\;
    $X \gets x$\;
    $N \gets n$\;
    \While{$N \neq 0$}{
        \eIf{$N$ is even}{
            $X \gets X \times X$\;
            $N \gets \frac{N}{2} $ \Comment*[r]{This is a comment}
        }{\If{$N$ is odd}{
                $y \gets y \times X$\;
                $N \gets N - 1$\;
            }
        }
    }
\end{algorithm}


\subsection{交叉引用}
被引用的地方需加上\textbackslash label\{\},要引用时使用\textbackslash ref\{\}:如图\ref{fig:example},表\ref{tab:example},算法\ref{alg:example}。


\section{结论}

更多\LaTeX 排版技巧,请参阅:\href{https://www.overleaf.com/learn/latex/Learn_LaTeX_in_30_minutes}{Learn LaTeX in 30 minutes}



\newpage
\bibliographystyle{acm}
\bibliography{reference}


\end{document}